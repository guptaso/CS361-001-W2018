\documentclass[12pt]{article} 

\usepackage{cite}

\title{The Missplaces, The Missplaces}
\author{Sonica Gupta (guptaso) and Lauren Sunamoto (sunamotl)} 

\begin{document}
\maketitle 


\section{intro}
	According to the Daily News, one in five adults misplaces personal belongings every week. We are included in that group. In fact, this is a very common occurance in our lives. Loosing our keys, walets, or worse, phone, misplacing items can be extremely stressful and a overall great burden. According to research, this occurance is not only emotionally taxing but also costly. Americans lose about six thousand dollars over a lifetime. This is equivalent to about a thousand Dutch Bros. 

\section{solution}
	Imagine a mobile application is available 24/7. It is reliable and can be trusted to keep track of all the things you value most. By just supplying a photo, this application will tag your location and time of documentation. It will be simple to quickly update the status of your items by adding another photo. The application will automatically organize this data by removing the prior location of the belonging. helping to keep the storage space minimal. The location of these items will be securely stored for easy access in the future. Not only will it give peace of mind, it will organize your life so you can focus on more important things such as your upcoming computer science assignment. 


\section{comparison} 
	You might be wondering, why not just use the Camera Roll already on your phone. But, this can be very costly in terms of storage space. Furthermore, it will be constantly reorganizing information to avoid not useful photos that take up space. It will also maintain this by periodically asking if you need to update or remove an item. Another similar approach is the Tile which can be attached to your belonging and tracked by your phone via bluetooth. This can drain your phone's battery and can be faulty in updating the item's location and longevity with some saying it was ineffective. Therer can also be hardware problems that are not applicable with our mobile application. 

\section{limitations}
	A limitation on this product is that it requires some user vigilence in providing updated photos. Although will send reminders to avoid this problem. Another limitation is that if the user loses their phone, we seek to back up this data that could be securely accessed anywhere.

\section{resources}
	We will require software that is needed for the creation of web and mobile application. Additionally, a smartphone for testing and possibly a database to backup user information.

\section{challenge}
	The most challanging task may arise when creating a database that will securely hold the sensitive information. To protect our users information, we will have to implement a way to avoid outsiders getting into private accounts. We will require the user to provide a username, password, and security questions which will allow us to determine if it is the user's account.  


\cite{dailyNews}

\bibliographystyle{plain}
\bibliography{ref}


\end{document}

