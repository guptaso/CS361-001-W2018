\documentclass[12pt]{article}
%\usepackage{times}%
%this is a comment
\title{Vision Statement}
\author{Fischer Jemison (jemisonf) and Sean Gillen (gillens)}




\begin{document}
\maketitle
\tableofcontents


\section{The Problem}
A frequent problem when working on markdown files in a github project is that conventional programming tools make it difficult to view what your file will look like when it is ultimately rendered on github. This is a problem that the authors have experienced and that others who have used github will likely have experienced as well.
\section{The Solution}
To solve this, we propose creating a c++ program that can convert markdown to pdf and html for easy viewing. This will provide a command line interface for doing conversions as well as a set of libraries to allow users to create their own programs for parsing markdown.
\section{Advantages of our Solution}
\begin{itemize}
	\item Speed: c++ is difficult to program in, so it is not the easiest tool to use for this problem, but will outperform tools made in other languages
	\item Modularization: the program will be broken into components that others can easily use
	\item Open Source: make it easier for others to use our code while lengthening the lifespan of the projecting by making it so that the original developers aren't solely responsible for maintaining the code.
\end{itemize}
\section{Technology Used}
\begin{itemize}
	\item c++ 
	\item pdf
	\item html/css
\end{itemize}
\section{Application Architecture Overview}
	\subsection{Core Components}
		\begin{itemize}
			\item .md file input: convert markdown to c++ data
			\item .pdf data output: convert c++ data to pdf data
			\item .html/.css output: convert c++ data to html or css
			\item file outputs: convert pdf/html/css data to an actual file
		\end{itemize}
	\subsection{Methodology}
		Each component will be encapsulated in a c++ library. Components will have methods to allow them to both communicate with other c++ objects and to output data to standard output so they can communcate with non-c++ programs.

\section{Stretch Goals}
	This section contains ideas to expand the project outside of the core architecture.
	\begin{itemize}
		\item Allow users to select custom styling
		\item Create different "modules" that fulfull different functions, like a program that takes markdown data from standard input and writes pdf data to standard output. Consider multiple potential use cases for the core architecture.
		\item Add output types other than pdf and html
	\end{itemize}
%\section{Plain Text}
%Hello, world!
%\subsection{Bold Text}
%{\bf Hello, world!}
%\subsubsection{Bold and Large Text}
%{\Large \bf Hello, world!!!}
%
%\section{Textbook}
%There is no required textbook. Here are some useful books on software engineering
%
%Software Engineering: Theory and Practice~\cite{pfleeger2010software}
%
%Software Engineering~\cite{sommerville2011software}
%
%
%\bibliography{myref}
%\bibliographystyle{plain}

\end{document}
